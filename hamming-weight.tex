\documentclass{beamer}

\usepackage{iansslides}

\begin{document}

\section{Hamming Weight}

\begin{frame}{Hamming distance}
  \begin{table}
    \centering
    \begin{tabular}{rcccccccc}
      \toprule
      $a$      & 0 & 1 & 1 & 0 & 1 & 1 & 0 & 1 \\
      $b$      & 1 & 0 & 1 & 0 & 0 & 1 & 1 & 1 \\
      \midrule
      $\oplus$ & 1 & 1 & 0 & 0 & 1 & 0 & 1 & 0 \\
      \bottomrule
    \end{tabular}
  \end{table}
  \begin{itemize}
    \item The Hamming distance between two words of equal length is the number of places in which they differ.
    \item The Hamming distance between $a$ and $b$ is 4.
  \end{itemize}
\end{frame}

\begin{frame}{Hamming weight}
  \begin{table}
    \centering
    \begin{tabular}{rcccccccc}
      \toprule
      $a$      & 0 & 1 & 1 & 0 & 1 & 1 & 0 & 1 \\
      $\bar{0}$      & 0 & 0 & 0 & 0 & 0 & 0 & 0 & 0 \\
      \midrule
      $\oplus$ & 0 & 1 & 1 & 0 & 1 & 1 & 0 & 1 \\
      \bottomrule
    \end{tabular}
  \end{table}
  \begin{itemize}
    \item The Hamming of a word is the number of non-zero symbols in it.
    \item The Hamming weight of $a$ is five.
    \item This is equal to the Hamming distance between $a$ and the zero word.
  \end{itemize}
\end{frame}


\begin{frame}{Loop-up tables}
  \begin{exampleblock}{Exercise}
    Write an algorithm that counts the number of bits set in an integer.
  \end{exampleblock}
  
  \pause
  
  You might try the following methods:
  \begin{itemize}
    \item Bit shifting
    \item Look-up table
    \item Kernighan's method
    \item popcount
  \end{itemize}
\end{frame}

\end{document}