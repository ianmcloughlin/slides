\documentclass{beamer}

\usepackage{iansslides}

\begin{document}

\section{Data}


\begin{frame}{Sending data via HTTP}
  \begin{description}
    \item[Sending] data is what HTTP is used for.
    \item[HTTP] sets out rules for sending data.
    \item[Rules] allow computers that know very little about each other to communicate without lengthy negotiations.
    \item[Sometimes] HTTP is not enough. We need more rules.
    \item[Building] on top of HTTP, we can send complex data --- not just HTML, CSS and JavaScript.
    \item[Commonly] these days we send images, videos, binary files, and even instances of objects.
  \end{description}
\end{frame}


\begin{frame}{Why send objects over HTTP?}
  \begin{description}
    \item[Databases] store a lot of the world's data.
    \item[Data] are usually stored using some sort of structure, so that they're reusable.
    \item[Object] usually means blueprint, the plans of the house.
    \item[Instance] usually means realisation of an object, like an actual house built from the plans.
    \item[JavaScript] doesn't make the distinction too clear.
    \item[Often] we would like to send a small chunk of structured data from a database server to a user's machine. Then their browser can display it nicely.
    \item[HTTP with JSON] can do this. 
  \end{description}
\end{frame}


\begin{frame}{JSON}
  \begin{description}
    \item[JavaScript] Programming language.
    \item[Object] Groups of name--value pairs.
    \item[Notation] Set of rules for representing objects.\\[1cm]
    \item[JSON is] just text --- text that conforms to a syntax.
    \item[Influenced] by JavaScript's syntax, but it is useable in all languages.
    \item[Represents] information in text form.
    \item[Popular] because it is easy to send over HTTP and parse in JavaScript.
  \end{description}
\end{frame}

\begin{frame}{Sending JSON}
  \tikzstyle{block} = [rectangle, fill=azure(colorwheel), text width=4.5em, text centered, minimum height=4em]
  \tikzstyle{line} = [draw, -latex']
  \tikzstyle{cloud} = [ellipse, fill=wildwatermelon, text width=5em, text centered]
  \tikzstyle{startstop} = [rectangle, rounded corners, text width=5em, minimum height=4em, text centered, fill=tiffanyblue]
  
  \begin{adjustbox}{max totalsize={.9\textwidth}{.6\textheight},center}
  
  \tikzstyle{rect} = [rectangle,fill=gmitblue,text width=4.5em,text centered,minimum height=4em,rounded corners,text=white]
  \tikzstyle{line} = [draw,->,very thick]
  \tikzstyle{oval} = [ellipse,fill=gmitred,text width=5em,text centered,text=white]
    
    \begin{tikzpicture}[node distance=4cm]
    \node [oval] (object1) {Object Instance};
    \node [rect, below of=object1] (json1) {JSON};
    \node [rect, right of=json1, node distance = 6cm] (json2) {JSON};
    \node [oval, above of=json2] (object2) {Object Instance};
    % Draw edges
    \path [line] (object1) -- node[style={rectangle,fill=white,draw}]{stringify} (json1);
    \path [line] (json1) -- node[style={rectangle,fill=white},draw ]{HTTP} (json2);
    \path [line] (json2) -- node[style={rectangle,fill=white,draw}]{parse} (object2);
    % Draw Memory
    \draw [color=gray, dashed](-2,-1.5) rectangle (2,1.25);
    \node at (-1.4,-1.35) [] {\tiny Machine 1};
    \draw [color=gray, dashed](4,-1.5) rectangle (8,1.25);
    \node at (4.6,-1.35) [] {\tiny Machine 2};
    \end{tikzpicture}
  \end{adjustbox}
\end{frame}

\begin{frame}[fragile]{JSON Example}
  \begin{minted}{json}
{
  "employees": [
    {"firstName":"John", "lastName":"Doe"},
    {"firstName":"Anna", "lastName":"Smith"},
    {"firstName":"Peter", "lastName":"Jones"}
  ]
}
  \end{minted}
\end{frame}


\begin{frame}[fragile]{Using JSON in JavaScript}
  The key here is that an object instance on one machine can be turned into JSON text, transmitted, and then reconstructed into an object instance on another machine. 
  \begin{minted}{javascript}
// Turning text into a JavaScript object.
var obj = JSON.parse(text);
// obj is an obect instance.

// Turning a JavaScript object into text.
var text = JSON.stringify(obj);
// text is a string.
  \end{minted}
\end{frame}

\begin{frame}{JSON Syntax}
  \begin{description}
    \item[Objects] identified by curly braces. \\
    \hspace{0.5cm} \mintinline{json}{{}}
    \item[Name:Value] pairs within objects separated by a comma. \\
    \hspace{0.5cm} \mintinline{json}{{"name": "Ian", "fingers": 10}}
    \item[Lists] identified by square brackets. \\
    \hspace{0.5cm} \mintinline{json}{["Ian", "Marco", "Sam"]}
    \item[Double quotes] around all names and strings. \\
    \hspace{0.5cm} \mintinline{json}{"Ian"}
  \end{description}
\end{frame}

\begin{frame}{JSON Types}
  \begin{itemize}
    \item Numbers \\
    \hspace{0.5cm} \mintinline{json}{123.456}
    \item Strings \\
    \hspace{0.5cm} \mintinline{json}{"Hello, world!"}
    \item Boolean \\
    \hspace{0.5cm} \mintinline{json}{true}
    \item Arrays\\
    \hspace{0.5cm} \mintinline{json}{[1,2,3]}
    \item Objects\\
    \hspace{0.5cm} \mintinline{json}{{"name": "Ian"}}
    \item null \\
    \hspace{0.5cm} \mintinline{json}{null}
  \end{itemize}
\end{frame}



\begin{frame}{eXtensible Markup Language}
  \begin{description}
    \item[XML] is a JSON alternative. 
    \item[eXtensible] Designed to accommodate change.
    \item[Markup] Annotates text.
    \item[Language] Set of rules for communication.
    \item[JSON] seems to be used more frequently than XML in most modern web applications.
    \item[XML] seems to be a little more verbose.
    \item[XML] was designed with more uses in mind.
    \item[HTML] and XML look similar.
  \end{description}
\end{frame}

\begin{frame}[fragile]{XML Example}
  \begin{minted}{xml}
<?xml version="1.0" encoding="UTF-8"?>
<book isbn-13="978-0131774292" isbn-10="0131774298">
  <title>Expert C Programming: Deep C Secrets</title>
  <publisher>Prentice Hall</publisher>
  <author>Peter van der Linden</author>
</book>
  \end{minted}
\end{frame}


\begin{frame}{Using XML}
  \begin{description}
    \item[Tags] are not pre-defined tag names -- you make them up yourself.
    \item[Syntax] follows a tree-like pattern. Tags can be nested within other tags.
    \item[Document Object Model (DOM)] is a concept related to XML.
  \end{description}
\end{frame}


\begin{frame}{XML Syntax}
  \begin{description}
    \item[Declaration] XML documents should have a single line at the start stating that they are XML, the version of XML used, and an encoding.
    \item[Elements] are the main form of structure in XML, and are enclosed in angle brackets.
    \item[Root element] XML must have a single root element that wraps all others.
    \item[Attributes] Elements can have attributes, which are name--value pairs within the angle brackets. A given attribute name can only be specified once per element.
    \item[Entity references] Certain characters must be escaped with entity references, e.g.\mintinline{html}{&lt;} for \mintinline{python}{<}.
    \item[Case sensitive] Everything in XML is case sensitive.
  \end{description}
\end{frame}

\begin{frame}[fragile]{XML Syntax Example}
  \begin{minted}{xml}
<?xml version="1.0" encoding="UTF-8"?>
<parent-element attribute-name="attribute-value">
  <child name="value">Text</child-element>
  <child name="value">Text</child-element>
  <child name="value">Text</child-element>
  <lone-warrior />
</parent-element>
  \end{minted}
\end{frame}

\begin{frame}{Document Object Model}
  \begin{description}
    \item[The DOM] is a programming interface for HTML and XML documents.
    \item[Models] the document as a structured group of nodes that have properties and methods.
    \item[Connects] web pages to scripts or programming languages.
    \item[Use document.createElement], document.createTextNode and document.element.appendChild to add to the DOM.
    \item[Use document.getElementById] to access elements of the DOM.
    \item[Extensively] used in web applications.
    \item[AngularJS, jQuery and React] have different ideas about the developers relationship with the DOM.
  \end{description}
  \citeurl{developer.mozilla.org/en-US/docs/Web/API/Document\_Object\_Model/Introduction}
\end{frame}


\begin{frame}{Using JSON and XML}
  \begin{description}
    \item[JSON and XML] don't do anything by themselves --- they're just text.
    \item[HTTP] can be used to retrieve JSON (or XML) from a server.
    \item[Developers] can use JavaScript to retrieve JSON (or XML) in the background of a web application.
    \item[The DOM] of the web page currently displayed in a browser can then be manipulated to display this information.
    \item[Web applications] are really just web pages that use this trick.
    \item[The mechanism] by which a web application does this is called AJAX.
    \item[Facebook] uses this to display new posts once the user has scrolled to the bottom of the screen.
  \end{description}
\end{frame}


\begin{frame}{Asynchronous JavaScript and XML}
  
  \begin{description}
    \item[AJAX] stands for Asynchronous JavaScript and XML.
    \item[Asynchronous] Doesn't block our other JavaScript code.
    \item[JavaScript] Programming language for the web.
    \item[XML] eXtensible Markup Language.
    \item[Page refreshes] happen when the webpage is removed from the view, and a new one is rendered.
    \item[JavaScript] can be used to change this behaviour. On clicking a link, and AJAX call can be triggered instead, and the web page remains.
    \item[Location bar] can even be manipulated.
  \end{description}
\end{frame}


\begin{frame}{More about AJAX}
  \begin{description}
    \item[AJAX] allows us to make a HTTP request from JavaScript, receive the response from that request and deal with it.
    \item[Despite] the name, we don't have to receive XML --- we can use JSON or anything else.
    \item[Calls] happen asynchronously --- other code can run while waiting.
    \item[HTTP requests] are usually relatively slow.
    \item[Callback functions] are used when HTTP transactions are complete.
    \item[JavaScript] libraries provide easy-to-use AJAX functions.
  \end{description}
\end{frame}


\begin{frame}[fragile]{Raw AJAX Example}
  \begin{minted}{javascript}
var xmlhttp = new XMLHttpRequest();

xmlhttp.onreadystatechange = function() {
  if (xmlhttp.readyState == 4) {
    var mydiv = document.getElementById("mydivid");
    mydiv.innerHTML = xmlhttp.responseText;
  }
};

xmlhttp.open("GET", "https://goo.gl/2GCplC");
xmlhttp.send();
  \end{minted}
\end{frame}


\begin{frame}{AJAX Example Explained}
  \begin{description}
    \item[XMLHttpRequest] is a built-in class that provides AJAX functionality in JavaScript.
    \item[onreadystatechange] should be set to a function to run every time something happens in our HTTP call.
    \item[open] is called to initialize the request.
    \item[send] is used to send the request to the server.
    \item[readyState] changes when the state of the AJAX call changes. This triggers a call to httpRequest.onreadystatechange.
  \end{description}
\end{frame}


\begin{frame}[fragile]{Using jQuery}
jQuery is much easier to use.
\hr
  \begin{minted}{html}
<script src="jquery.min.js"></script>
  \end{minted}
  \begin{minted}{javascript}
$.get("https://goo.gl/2GCplC", function(data) {
  $("#mydivid").html(data);
});
  \end{minted}
\end{frame}

\end{document}