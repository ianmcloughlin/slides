\documentclass{beamer}

\usepackage{iansslides}

\begin{document}

\section{Non-deterministic Turing Machines}

\begin{frame}{Non-deterministic Turing machine}
  \begin{description}
    \item[The usual] Turing machines are often called deterministic Turing machines.
    \item[Deterministic] Turing machines have exactly one row in their state table for every combination of (non-terminal) state and tape symbol.
    \item[This means] there is only one path to follow at a given point in time.
    \item[Nondeterministic] Turing machines can have any number of rows for each state/symbol (including none).
    \item[Essentially] they allow for parallel computation -- they can branch into two or more paths at the same time.
  \end{description}
\end{frame}


\begin{frame}{Non-deterministic Turing machine and languages}
  \begin{description}
    \item[Languages] are accepted by non-deterministic Turing machines, where an input string is accepted if any branch ends in the accept state.
    \item[Deciders] -- if a non-deterministic Turing machine always halts on all branches of computation, no matter what the input, then we say it decides the language it accepts.
    \item[Any] language that is accepted (or decided) by a non-deterministic Turing machine has some deterministic Turing machine that accepts (or decides) it. So non-deterministic Turing machines don't really have any extra abilities over deterministic ones.
  \end{description}
\end{frame}


\begin{frame}{Non-deterministic polynomial time}
  \begin{definition}
    A decision problem is in the NP complexity class if it is decidable by a non-deterministic Turing Machine in polynomial time.
  \end{definition}
  
  \begin{alertblock}{P is a subset of NP}
    Note that every determinisitic Turing machine is also a non-deterministic one, by our definitions.
    The P complexity class is a subset of NP because of this.
  \end{alertblock}

  \begin{alertblock}{Equivalent definition}
    An equivalent definition of NP that you may come across is that NP is the set of languages $A$ that can be verified in polynomial time.
    By verified we mean that a deterministic Turing machine can accept a language $\{ wc \}$ where $w$ is in $A$ and $c$ is some string, called the certificate for $w$.
  \end{alertblock}

\end{frame}


\begin{frame}{NP-complete problems}
  \begin{definition}
    A problem is NP-hard if each problem in NP can be reduced to it in polynomial time.
  \end{definition}

  \begin{alertblock}{Reduction}
    Reduction is a way of converting one problem into another, so that a solution to one is a solution to the other.
    By reducing decision problem A to decision problem B, we mean that we can transform inputs to A into inputs to B in such a way that a given input to A is accepted iff the corresponding input to B is.
  \end{alertblock}

  \begin{definition}
    A problem is NP-complete if it's in NP and is NP-hard.
  \end{definition}
\end{frame}





\end{document}